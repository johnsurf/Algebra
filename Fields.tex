\section{Fields}

\begin{theorem}{(Fields have no non-trivial Ideals):}
If a field $F$ has an ideal $I$, then $I = (0)$ or $I = (1) = F$. 
\end{theorem}
\begin{proof}
Let $a \in I$ and  $a\ne 0$. Then $a a^{-1} = 1 \in I$ and so $f.1 = f \in I$ for any $f\in F$ Therefore $ F \subset I$. But $I \subset F$ by definition of Ideals. Therefore $ I = F$. The only other possibility is $a=0$ in which case we have the trivial Ideal $ I = (0) = 0$.
\end{proof}

If $F$ is any field, then the smallest subfield of $F$ that contains the identify element 1 is called the {\elevenit prime subfield}\/ of $F$. If $F$ is a finite field, then its prime subfield is isomorphic to $\Z_p$, where $p = \char(F)$ for some prime $p$. When we want to emphasize $\Z/p\Z$ with its field structure, we denote it by $\F_p$. The prime subfield of $\Z_p$ which is isomorphic to $\Z/p\Z$ is $\F_p$. The only prime fields are the fields of residues modulo $p$ and the field of rationals.\\

The prime subfield of a field $F$ is the intersection of all subfields of $F$. The intersection of any two subfields is a field and similarly for all the other subfields. Also the intersection of all subfields has no proper subfields, since any such would be subfields of R and hence must contain the intersection which is impossible. Hence the intersection is a prime subfield. $F$ cannot contain two distinct prime subfields, since their intersection would also be a subfield and hence must be identical with both. 
