\section{Resultant of Two Polynomials}


\section{Resultant Expressed as a Sylvester Determinant}

Consider two polynomials 
\begin{align*}
f(x)  &= \sum_{i=0}^n f_i\,x^{n-i} = f_0 x^n + f_1 x^{n-1} + f_2 x^{n-2} + \hdots + f_{n-1} x + f_n\\
g(x) &=  \sum_{j=0}^m g_j\,x^{m-j} = g_0 x^m + g_1 x^{m-1} + g_2 x^{m-2} + \hdots + g_{m-1} x + g_m\\
\end{align*}
Let use assume that $f(x)$ and $g(x)$ have a common factor $(x-\alpha)$. Then 
\begin{align*} f(x) &= (x-\alpha) f_1(x) \quad \hbox{and} \\
                     g(x) & = (x-\alpha) g_1(x).\end{align*}
Then $$f(x) g_1(x) = g(x) f_1(x) = {f(x) g(x) \over x-\alpha}.$$
 
 Assuming that 
\begin{align*}
f_1(x)  &= -\sum_{i=0}^{n-1} z_i\,x^{n-1-i} = -(z_0 x^{n-1}  + z_1 x^{n-2}  + z_2 x^{n-3}  + \hdots  +z _{n-2} x + z_{n-1} ) \\
g_1(x) &= \sum_{j=0}^{m-1} y_j\,x^{m-1-i} =  y_0 x^{m-1} + y_1 x^{m-2} + y_2 x^{m-3} + \hdots  + y_{m-2} x + y_{m-1} \\
\end{align*}
By equating like powers of $x$, we can write out a linear system corresponding to the equation $$f(x) g_1(x) - g(x) f_1(x) = 0$$ we find 
\begin{align*}
-g(x) f_1(x)  = ~ &g_0 z_0 x^{n+m-1}  + g_0 z_1 x^{n+m-2}  + g_0 z_2 x^{n+m-3}  + \hdots  + g_0 z_{n-2} x^{m+1} + g_0  z_{n-1}x^m + \\
                      &g_1 z_0  x^{n+m-2}  + g_1 z_1 x^{n+m-3}  + \hdots  + g_1 z_{n-3} x^{m+1} + g_1 z_{n-2} x^{m} + g_1 z_{n-1} x^{m-1}  +\\
                      &g_2 z_0  x^{n+m-3}  + g_2 z_1 x^{n+m-4}  + \hdots  + g_2 z_{n-3} x^{m} + g_2 z_{n-2} x^{m-1} + g_2 z_{n-1} x^{m-2}  +\\
                      &\vdots  ~~~+ \\
                      &\hdots  + (g_{m-2} z_{n-1} + g_{m-1} z_{n-2} + g_{m} z_{n-3} ) x^{2} + (g_{m-1} z_{n-1}  +  g_m z_{n-2} )x + g_m z_{n-1}\\
\end{align*}
\begin{align*}
g(x) g_1(x)  = ~ &f_0 y_0 x^{n+m-1}  + f_0 y_1 x^{n+m-2}  + f_0 y_2 x^{n+m-3}  + \hdots  + f_0 y_{m-2} x^{n+1} + f_0  y_{n-1}x^n + \\
                      &f_1 y_0  x^{n+m-2}  + f_1 y_1 x^{n+m-3}  + \hdots  + f_1 y_{n-3} x^{n+1} + f_1 y_{m-2} x^{n} + f_1 y_{n-1} x^{n-1}  +\\
                      &f_2 y_0  x^{n+m-3}  + f_2 y_1 x^{n+m-4}  + \hdots  + f_2 y_{n-3} x^{n} + f_2 y_{m-2} x^{n-1} + f_2 y_{n-1} x^{n-2}  +\\
                      &\vdots  ~~~+ \\
                      &\hdots  + (f_{n-2} y_{m-1} + f_{n-1} y_{m-2} + f_{n} y_{m-3} ) x^{2} + (f_{n-1} y_{m-1}  +  f_n y_{m-2} )x + f_n y_{m-1}\\
\end{align*}

In order to satisfy $f(x) g_1(x) - g(x) f_1(x) = 0$, all the terms multiplying like powers of $x$ each must vanish. This can be arranged as a linear system in the vector $[ z_0,z_1,..., z_{n-1},y_0,y_2,...y_{m-1}]$. The coefficient matrix of this linear system is equal to the transpose of the Sylvester Resultant matrix which is defined as
\begin{align*}
R_{m,n}(g,f)=
\underbrace{
%\rdelim\}{4}{6mm}[$J$] \\ \\ \\[4mm]  \rdelim\}{3}{6mm}[$H$] \\ \\
\begin{bmatrix}
g_0 & \cdots & g_m & & \\
&\ddots & \cdots & \ddots & \\
& & g_0 & \cdots & g_m \\
f_0 & \cdots & ~~~~~~f_n & & \\
&\ddots & \cdots & \ddots & \\
& &f_0 & \cdots &f_n \\
\end{bmatrix}
}_{m+n}
\begin{array}{l}
  \\[-8mm] \rdelim\}{3.7}{6mm}[$n$ lines] \\ \\ \\[4mm]  \rdelim\}{3}{6mm}[$m$ lines] \\ \\
\end{array} \\[-1ex]
\end{align*}

The corresponding linear system that sets all the coefficients of the same powers of $x$ to $0$ in $f(x) g_1(x) - g(x) f_1(x) = 0$ is 
\[
\begin{bmatrix}
g_0 & \cdots & g_m & & \\
&\ddots & \cdots & \ddots & \\
& & g_0 & \cdots & g_m \\
f_0 & \cdots &~~~~~~f_n & & \\
&\ddots & \cdots & \ddots & \\
& &f_0 & \cdots &f_n \\
\end{bmatrix}^T 
\begin{bmatrix}
z_0\\
\vdots\\
z_{n-1}\\
y_0\\
\vdots \\
y_{m-1}\\
\end{bmatrix} = 
\begin{bmatrix}
0\\
\vdots\\
0\\
0\\
\vdots\\
0\\
\end{bmatrix}
\]

The criterion for a non-trivial solution to the above linear system is 
\[
\det \begin{bmatrix}
g_0 & \cdots & g_m & & \\
&\ddots & \cdots & \ddots & \\
& & g_0 & \cdots & g_m \\
f_0 & \cdots & ~~~~~~f_n & & \\
&\ddots & \cdots & \ddots & \\
& &f_0 & \cdots &f_n \\
\end{bmatrix} = 0\]
The rank of the matrix $R_{m,n}(g,f)$ also gives the multiplicity of the common root of $f(x)$ and $g(x)$.
The leading term of this determinant is $g_0^n f_n^m$. The terms of in the expansion an $(n + m) \times (n+m)$ matrix $M$  of the form
$\sum_{\sigma}\epsilon_{r_1,r_2,r_3,\hdots r_{n+m}} m_{1r_1} m_{2r_2} m_{3r_3}  \hdots m_{(n+m) r_{n+m}} $ where the sum is over all permutations $\sigma$ of the numbers $[1,2,\hdots,n+m]$ and 
$sgn(\sigma) = \epsilon_{r_1,r_2,r_3,\hdots r_{n+m}} $ is the sign of the permutation. In the case of the Sylvester Resultant the terms take the form 
\[sgn(\sigma) g_0^{\nu_0}g_1^{\nu_1}\hdots g_m^{\nu_m} f_0^{\mu_0} f_1^{\mu_1} \hdots f_n^{\mu_n},\]
where 
\begin{align*}
n &=  \nu_0 + \nu_1   +   \nu_2 + \hdots  + \nu_m \\ 
m & =  \mu_0 + \mu_1  +  \mu_2 + \hdots  + \mu_n \\
\end{align*}
Factoring out $g_0^n f_0^m$ the generic term in the expansion of the determinant becomes
\[sgn(\sigma) g_0^n f_0^m  ( {g_1\over g_0})^{\nu_1} ({g_2\over g_0})^{\nu_2} \hdots ({g_m \over g_0})^{\nu_m}  ({f_1\over f_0})^{\mu_1} ({f_2\over f_0})^{\mu_2} \hdots ({f_n\over f_0})^{\mu_n} \]
Hence the Sylvester Resultant has the form 
\[ R_{m,n}(g,f) = g_0^n f_0^m \phi(\alpha,\beta) \] where $\phi$ is an integral function of the roots $\alpha_i$ and $\beta_i$.
As the determinant vanishes for any common roots $\alpha_i = \beta_k$, the determinant is divisible by $(\alpha_i - \beta_k)$ and the determinant is divisible by 
\[ g_0^n f_0^m \prod_{i,k} (\alpha_i - \beta_k) \]

\begin{theorem} $R_{m,n}(g,f)$ is a homogenous polynomial with integer coefficients $g_i,f_j$.
\begin{itemize}
\item[]{(i)} $R_{m,n}(g,f)$ is homogenous of degree $n$ in $g_m,...,g_0$ and degree $m$ in $f_n,...,f_0$.
\item[]{(ii)}If $g_i$ and $f_i$ are regarded as having degree $i$. then $R_{m,n}(g,f)$ is homogenous of degree $mn$.
\end{itemize}
\end{theorem}
\begin{proof}
It is obvious that $R_{m,n}(g,f)$ is a homogenous polynomial with integer coefficients in the coefficients $g_i, f_j$, of total degree $m+n$.
Moreover, to replace $g_i$ by $tg_i$ and $f_j$ by $uf_j$ in the Sylvester matrix means that we multiply each of the first $m$ rows by $t$
and each of the last $n$ rows by $u$, and thus the determinant $R_{m,n}(g,f)$ by $t^mu^n$, which shows part ({\it{i}}). It is here best to 
treat $g_i, f_j, t$ and $u$ as different inderminants a  do the calculations in $F(g_0,...,g_m, f_0,...,f_n, t,u)$.\\
Similarly, ({\it ii}) follows because to replace each $g_i$ by $t^ig_i$ and each $f_j$ by $t^jf_j$ in $R_{m,n}(g,f)$ yields the same result as 
multiplying the $i$th row by $t^{m+i}$ for $i=1,2,...,m$ and by $t^i$ for $i=m+1,...,m+n$, and the $j$th column by $t^{-j}$\/. The overall 
effect it to multiply the determiant $R_{m,n}(g,f)$ by $t^K$ where $K = mn+\sum_{i=1}^{m+n} i - \sum_{j=1}^{m+n} j = mn$.\qed 
\end{proof}

%\begin{bmatrix}
%z_0 \\
%z_1 \\
%\vdots \\
% z_{m-1} \\
%y_0 \\
%\vdots\\
%y_{n-1} \\
%\end{bmatrix}^T 
%= 
%\begin{bmatrix}
%0 \\
%0 \\
%\vdots \\
%0\\
%0 \\
%\vdots\\
%0 \\
%\end{bmatrix}
